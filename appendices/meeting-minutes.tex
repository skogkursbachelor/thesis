\chapter{Meeting Minutes}
\label{appendix:meeting_minutes}
\section*{Møte med Veileder: 2025-01-14}
\begin{itemize}
\item
  Kjernetid: 9-15, 6 timer hver dag. Vi har forelesning på mandager
  14-16 og onsdager 10-12, og gruppearbeid i samme fag som vil ta litt
  av kjernetiden første ukene.
\item
  Møte: Ukentlig med veileder, og annen hver uke med oppdragsgiver?
\item
  Ha klart spørsmål om kravspesifikasjon til Dag
\end{itemize}

\section*{Møte med Veileder og
Produkteier: 2025-01-21}

\subsection*{Agenda}
\begin{itemize}
\item
  Referere til andre sine bacheloroppgaver i prosjektplan?
\item
  Flere MVPer? Vi har nå tenkte 1 MVP midt i mars for user-testing
\item
  Få en gjennomgang til av
\item
  Annen data og utregning
\item
  NASA SMAP og ESA Sentinel-1
\item
  Hvilke data skal brukes
\item
  Hvordan få prognose frem i tid, bruke verktøy som allerede finnes
  eller lage selv?
\item
  Prosjektplan
\item
  Constraints
\item
  Delimitation (er Norge for stort område å fokusere på)
\end{itemize}

\begin{enumerate}
\item
  Dag
\item
  Kanskje Dag kan si litt om oppgaven til Peter
\item
  Gå gjennom det vi har hittil (prosjektplan, gantt)
\item
  Spørsmål
\end{enumerate}

\subsection*{Notater}
\textbf{Product \& Impact Goals:} 
\begin{itemize}
    \item Ligger på linje med Skogkurs'
forventninger
\item Impact (reduced uncertainty)
\end{itemize}
\textbf{Brukere:} 
\begin{itemize}
    \item \emph{Transportleder}
    \item Sjåfører
    \item ``Personer med kunnskap i feltet''
\end{itemize}
\textbf{Prognose:} 1-2 uker\\
\textbf{Fargekoding av vei (trafikklys):} Rød, gul, grønn\\
\textbf{Område:} Begrenset område - Gjøvik, steder det faktisk kan
bruke. For mye data.\\
\textbf{Bruk av data / variabler (begrensning):}
\begin{itemize}
    \item (Ikke konkret rett valg av data, mål å få testet at et produkt fungerer)
    \item \emph{Løsmasser}
    \item Markfuktighet
    \item Teledyp
    \item Grunnvannstilstand
    \item Skogbilveier
    \item Variere for forskjellige områder hvor det er relevant
    \item Evt. snitt data hvor flere lag brukes
    \item (ikke tenk på vekt av last)
\end{itemize}

\textbf{Brukertesting:} \\
\begin{itemize}
    \item Tidlig Wireframe slik at for teste tidlig versjon
    \item For transportledere, hjelp fra Skogkurs
\end{itemize}
\textbf{Begynne med å definere data som trengs og skal brukes i februar}
\begin{itemize}
    \item Nibio skogsportal (WMS)
    \item Traktor og skogsbilvei
    \item senorge - raster
\end{itemize}


\section*{Møte med Veileder: 2025-01-28}

\subsection*{Agenda:}
\begin{itemize}

\item
  Referere til andre sine bacheloroppgaver i prosjektplan?
\item
  Hvor spesifikt skal produktmål være?
\item
  I avsnitt om scrum, skal vi skrive om hvorfor vi ikke valgte de andre
  metodene?
\item
  Scope: burde man nevne data fra rapportene til Dag Fjeld?
\item
  Gjennomgang av scope, domain og task desc.
\end{itemize}

\subsection*{Notat:}

\begin{itemize}
\item
  Fokus på rapport, få med det som gikk bra, og hva som var utfordrende
\item
  Kartlegging i februar (data)
\item
  Finne kriteriene for data som skal eller kan brukes for å
  tilfredsstille krav
\item
  Se gjennom rapportene til D. Fjeld (abstrakt, introduksjon,
  konklusjon)

  \begin{itemize}
  \item
    Se om andre rapporter er referert
  \end{itemize}
\item
  Hvor mange uker satt av for rapport?:

  \begin{itemize}
  \item
    Kreves litt mer tid, men finn balanse
  \end{itemize}
\item
  Få teksten lest over av noen eksterne (andre studenter, familie?)

  \begin{itemize}
  \item
    Sette av litt tid for finpussing da for at noen skal lese over
  \item
    Passe på å ha en ``rød tråd'' gjennom rapporten, tenke på
    overgangene mellom deler av rapporten
  \end{itemize}
\item
  Kanskje finne lignende bachelor
\item
  Referer til andre oppgaver (ikke kopier rett fra teksten)
\item
  Skrive om SDLC

  \begin{itemize}
  \item
    Vise at man vet om andre metoder
  \item
    men ikke skrive for langt
  \item
    Finne argumenter for valg man har tatt, også andre steder i teksten
  \end{itemize}
\item
  Scope:

  \begin{itemize}
  \item
    Skogsdrift, tungtransport
  \item
    Kart
  \end{itemize}
\item
  Task description:

  \begin{itemize}
  \item
    Steg for å nå målet
  \item
    Analyse av kriterier som trengs for å nå mål
  \item
    Implementasjon av kart og webside
  \end{itemize}
\item
  Impact Goals

  \begin{itemize}
  
  \item
    Kanskje ha med noe om bærekraft, miljø, etc.
  \end{itemize}
\end{itemize}

\section*{Møte med Veileder: 2025-02-04}

\subsection*{Notat}
\begin{itemize}
\item
  Spørre produkteier om dataressurser vi har funnet
\end{itemize}

\section*{Møte med Produkteier: 2025-02-10}

\subsection*{Spørsmål}

\begin{itemize} 
\item
  OpenMeteo Ensemble API

  \begin{itemize}
  \item
    Hvor dyp jordfuktighet er nødvendig?
  \item
    Hvor nøyaktig er OpenMeteo?
  \end{itemize}
\item
  Detaljert info for skogsveg fra NIBIO: \\
  https://kilden.nibio.no/skogsbilveger\_ws/skogsbilveiInfo?sveiid=3407-14-1
\end{itemize}

\subsection*{Notat}

\begin{itemize}
\item
  Når vegen ble bygd: ØKS? Landbruksdirektoratet

  \begin{itemize}
  
  \item
    Kategorisere vegene i klasser
  \item
    Nyere og bedre veier etter 2015
  \end{itemize}
\item
  Hva er vegen bygd av
\item
  Forskjellige høydenivå på veger
\item
  Hvordan bruke soil moisture

  \begin{itemize}
  \item
    Sette klasser
  \end{itemize}
\item
  Grenser for data

  \begin{itemize}
  \item
    Kanskje ha dynamiske grenser (som kan endres av bruker)?
  \item
    Sette grenser ved bruk av historisk data?
  \end{itemize}
\item
  Ha med historisk data, da OpenMeteo kanskje ikke støtter dette.
\end{itemize}

\section*{Møte med Veileder: 2025-02-11}

\subsection*{Agenda}

\begin{itemize}
\item
  Snakke om samtalen med Dag Fjeld
\item
  Plakat/rapportmal 
\end{itemize}

\subsection*{Notat}
\begin{itemize}
    \item
      Skal spørre Pål om rapportmal
    \item
      Notere ned detaljer allerede for å ha med i den endelige rapporten
    \item
      Får mal på lynkurs, kan også finne på nett
    \item
      Info blir også gitt om plakat senere på seminar om presentasjon
\end{itemize}

\section*{Møte med Veileder: 2025-02-20}

\subsection*{Agenda}

\begin{itemize}
\item
  Vise fremgang av webside
\end{itemize}

\subsection*{Notat}

\begin{itemize}
    \item Husk å notere vanskeligheter underveis for å ha med i rapporten senere
\end{itemize}

\section*{Møte med Veileder: 2025-02-24}

\subsection*{Agenda}
\begin{itemize}
\item
  Vise fremgang av webside før møte med produkteier
\end{itemize}


\section*{Møte med Produkteier: 2025-02-25}

\subsection*{Agenda}

\begin{itemize}
\item
  Vise frem proto-prototype av webside
\end{itemize}
\subsection*{Notat}
\begin{itemize}
\item
  Copernicus, evt. prognose
\item
  Få implementert open-meteo soil moisture / soil temperature
\item
  Teledyp grenser:

  \begin{itemize}
  \item
    Både 0-7cm og 7-28cm jordtemperatur er null = grønt lys
  \item
    Jordfuktighet forskjell mellom forskjellige jordarter
  \end{itemize}
\end{itemize}

\section*{Møte med Veileder: 2025-03-04}

\subsection*{Agenda}

\begin{itemize}
\item
  Snakke om møte med produkteier forrige uke
\item
  Hva planen er fremover for denne sprinten
\end{itemize}

\subsection*{Notat}

\begin{itemize}
\item
  For klassifisering av skogsbilveg

  \begin{itemize}
  \item
    Se skiforeningens side
  \end{itemize}
\item
  Få med utfordringer, blindveier i prosessen etc. i rapport
\end{itemize}

\section*{Møte med Veileder: 2025-03-13}

\subsection*{Agenda}

\begin{itemize}
\item
  Fullt fokus på bacheloroppgaven fra nå
\end{itemize}

\subsection*{Notat}

\begin{itemize}
\item
  Se rapportskrivings krasjkurs
\item
  Rapportskrivingskurs 27. Mars
\item
  Første utkast av rapport til påske f.eks. tirsdag 8. april, gi til
  veileder for tilbakemelding
\item
  Tilbakemelding på utkast etter påske tirsdag 22. april
\end{itemize}

\section*{Møte med Veileder: 2025-03-18}

\subsection*{Agenda}
\begin{itemize}
\item
  Forside til rapporten, skal vi ha med egen eller blir den generert ved
  innlevering
\item
  Eksempler på ``A''-rapporter
\item
  Hva burde være med av kapitler, særlig introduksjon. F.eks. skal ikke
  ha med gruppebakgrunn eller noe personlig om gruppen (ifølge Hjelmås)
\item
  Mellomrom eller innrykk mellom avsnitt
\item
  Referanse har tomt parantes hvis man ikke legger til dato for når
  siden ble sist oppdatert. Er dette nødvendig og hvis det ikke er
  oppgitt kan man bruke ``n.d.'' (no date)
\item
  Chapter vs.~Section. Virker litt unaturlig at det står Chapter 1, 2, 3
  osv.
\item
  Latex tar med forfatter og tittel på hver partall side, f.eks.
  ``erbj\&simonhou@NTNU: Lumber transport'' er dette meningen?
\item
  Er det relevant å ta med som utfordring at tidsfordeling med et annet
  fag på siden av bachelor
\end{itemize}
  
\subsection*{Notat}
\begin{itemize}
\item
  Kan ta med i rapporten at man finner frem til løsninger ved å sparre
  med AI.
\item
  Forside til rapporten, skal vi ha med egen eller blir den generert ved
  innlevering

  \begin{itemize}
  \item
    \emph{Får svar før innlevering.}
  \end{itemize}
\item
  Eksempler på ``A''-rapporter

  \begin{itemize}
  \item
  \end{itemize}
\item
  Hva burde være med av kapitler, særlig introduksjon. F.eks. skal ikke
  ha med gruppebakgrunn eller noe personlig om gruppen (ifølge Hjelmås)

  \begin{itemize}
  \item
    \emph{Hvis man skal ta det med så hold det kort.}
  \item
    \emph{Er mulig å få A selv om man tar med.}
  \end{itemize}
\item
  Mellomrom eller innrykk mellom avsnitt

  \begin{itemize}
  \item
    \emph{Mellomrom}
  \end{itemize}
\item
  Referanse har tomt parentes hvis man ikke legger til dato for når
  siden ble sist oppdatert. Er dette nødvendig og hvis det ikke er
  oppgitt kan man bruke ``n.d.'' (no date)?

  \begin{itemize}
  
  \item
    \emph{Prøv å fjern parentesen hvis ingen dato er oppgitt istedenfor
    å bruke n.d. eller no date}
  \end{itemize}
\item
  Chapter vs.~Section. Virker litt unaturlig at det står Chapter 1, 2, 3
  osv.

  \begin{itemize}
  
  \item
    \emph{Ja, chapter skal ytterst}
  \end{itemize}
\item
  Latex tar med forfatter og tittel på hver partall side, f.eks.
  ``erbj\&simonhou@NTNU: Lumber transport'' er dette meningen?

  \begin{itemize}
  
  \item
    \emph{At kapittel står er fint, men må ikke ha med forfatter
    og/eller tittel.}
  \end{itemize}
\item
  Er det relevant å ta med som utfordring at tidsfordeling med et annet
  fag på siden av bachelor. Også med tidsfordeling om at vi kun er 2 i
  gruppen.

  \begin{itemize}
  
  \item
    \emph{Hvis man kan ta med på en bra måt så ta med på en objektiv
    måte, ikke for spesifikt.}
  \item
    \emph{Ikke nevn følelser (adjektiver), ikke spesifiser 2 i gruppen
    (sensor vet det). Heller nevn at man var litt for optimistisk på
    tiden man hadde.}
  \item
    \emph{Trenger kanskje ikke nevne hvis det ikke ble lovet i planen
    eller at det ikke var et krav fra produkteier.}
  \end{itemize}
\item
  Ha tekst for hvert nye kapittel (som forklarer innholdet for
  kapitellet). Hvis man er i et senere kapittelet kan denne setningen
  gjerne inneholde et slags\\
\item
  Vær flink til å ha med figurer for å forklare prinsipper eller hvordan
  produktet fungerer, samt selve prossesen av prosjektet.

  \begin{itemize}
  \item
    Figurer er også fint å bruke i presentasjonen
  \end{itemize}
\item
  Universell Utforming burde tas med, f.eks. ha med spørsmål på
  brukertesting.

  \begin{itemize}
  \item
    Hvis man ikke finner en perfekt løsning for f.eks. design, så nevn
    det i rapport uansett. Eks. fargeblindhet
  \end{itemize}
\end{itemize}