\chapter{Superficial Deposits Code Values}
\label{appendix:superficial_deposit_codes}
\begin{longtable}{|p{3.5cm}|p{6.2cm}|c|}
    \hline
    \textbf{Navn} & \textbf{Beskrivelse} & \textbf{Kodeverdi}\\ 
    \endfirsthead
    
    \hline
    \textbf{Navn} & \textbf{Beskrivelse} & \textbf{Kodeverdi}\\ 
    \endhead
    \hline
    
    Løsmasser / berggrunn under vann, uspesifisert & Brukes for en avsetning der genetisk opprinnelse ikke er påvist, og det er heller ikke bestemt  om sedimentet er av marin opprinnelse. & 1 \\ \hline
    Morenemateriale, uspesifisert & Materiale plukket opp, transportert og avsatt av isbreer. Det er vanligvis, dårlig sortert og kan inneholde alt fra leir til stein og blokk. Mektighet, morenetype og overflateform kan variere. Benyttes ved kartframstilling i svært små målestokker. & 10 \\ \hline
    Morenemateriale, sammenhengende dekke, stedvis med stor mektighet & Materiale plukket opp, transportert og avsatt av isbreer, vanligvis hardt sammenpakket, dårlig sortert og kan inneholde alt fra leir til stein og blokk. Moreneavsetninger med tykkelse fra 0,5 m til flere ti-talls meter. Det er få eller ingen fjellblotninger i området. & 11 \\ \hline
    Morenemateriale, usammenhengende eller tynt dekke over berggrunnen & Materiale plukket opp, transportert og avsatt av isbreer. Det er vanligvis hardt sammenpakket, dårlig sortert og kan inneholde alt fra leir til stein og blokk. Områder med grunnlendte moreneavsetninger/hyppige fjellblotninger. Tykkelsen på avsetningene er normalt mindre enn 0,5 m, men den kan helt lokalt være noe mer. & 12 \\ \hline
    Moreneleire & Morenemateriale med særlig høyt leir- og siltinnhold, oftest meget kompakt. & 13 \\ \hline
    Avsmeltningsmorene (Ablasjonsmorene) & Hauger og rygger med løst lagret, delvis vannbehandlet og noe sortert morenemateriale avsatt fra stagnerende breer (dødis). Terrenget er preget av haug- og ryggformer med vekslende orientering. & 14 \\ \hline
    Randmorene / randmorenebelte & Rygger eller belter av morenemateriale som er skjøvet opp foran brefronten. Materialet er usortert og inneholder alle kornstørrelser fra leir til blokk. Noen steder kan morenematerialet finnes i veksling med breelvmateriale. & 15 \\ \hline
    Drumlin & Langstrakt, rettlinjet morenerygg dannet langs isbevegelsesretningen i bunnen av en bre. Ofte stor tykkelse, avrundet form og lengden kan være opp til noen km. & 16 \\ \hline
    Rogenmorene & Rygger av morenemateriale, orientert på tvers av brebevegelsen. & 17 \\ \hline
    Breelvavsetning (Glasifluvial avsetning) & Materiale transportert og avsatt av breelver. Sedimentet består av sorterte, ofte skråstilte lag av forskjellig kornstørrelse fra fin sand til stein og blokk. Breelvavsetninger har ofte klare overflateformer som terrasser, rygger og vifter. Mektigheten er ofte flere ti-talls meter. & 20 \\ \hline
    Breelv- og elveavsetning & Materiale transportert og avsatt av elver eller breelver. Sedimentet består av sorterte lag av forskjellig kornstørrelse fra fin sand til grus og stein. Det er ikke skilt mellom breelv- og elveavsetninger. Brukes kun i spesielle tilfeller. & 21 \\ \hline
    Ryggformet breelvavsetning (Esker) & Sortert og lagdelt materiale, vesentlig sand og grus, avsatt i tunneler eller sprekker i breen. Der avsetningen er stor nok til å danne figur på kartet brukes fargen for breelvavsetninger til å angi utbredelsen og eskersymbolet til å angi ryggformen. & 22 \\ \hline
    Haugformet breelvavsetning (Kame) & Materiale avsatt av smeltevann i hulrom i breen. Store avsetninger gis fargen for breelvavsetninger i kombinasjon med symbol for kame. & 23 \\ \hline
    Bresjø-/eller brekammeravsetning (Glasilakustrin avsetning) & Finkornig materiale avsatt i bresjø eller vannfylt brekammer hvor tykkelsen er mer enn 0,5 m og arealdekningen er stor nok til å danne figur på kartet. Mektigheten kan være flere ti-talls meter. & 30 \\ \hline
    Breelv- og bresjø-/brekammeravsetning (Glasifluvial og glasilakustrin avsetning) & Materiale avsatt av breelv eller i bredemte sjøer/brekammer. Det er ikke skilt mellom breelv- og bresjø-/kammeravsetninger. & 31 \\ \hline
    Innsjøavsetning (Lakustrin avsetning) & Materiale avsatt i innsjøer hvor tykkelsen er mer enn 0,5 m. & 35 \\ \hline
    Bresjø-/brekammer og innsjøavsetning (Glasilakustrin og lakustrin avsetning) & Benyttes hvis en ønsker å slå sammen de to avsetningstypene. I tilfelle brukes ikke separate farger for bresjø og innsjø på det samme kartbladet. & 36 \\ \hline
    Strandavsetning, innsjø og/eller bresjø & Strandvaskede sedimenter med mektighet større enn 0,5 m, dannet ved bølgeaktivitet i ferskvann. Materialet er ofte rundet og godt sortert. Kornstørrelsen varierer, men sand og grus er vanligst. & 37 \\ \hline
    Hav- og fjordavsetning, uspesifisert & Benyttes ved kartframstilling i svært små målestokker der en ikke skiller etter mektighet. & 40 \\ \hline
    Hav- og fjordavsetning, sammenhengende dekke, ofte med stor mektighet & Finkornige, marine avsetninger med mektighet fra 0,5 m til flere ti-tall meter. Avsetningstypen omfatter også skredmasser fra kvikkleireskred, ofte angitt med tilleggssymbol. Det er få eller ingen fjellblotninger i området. & 41 \\ \hline
    Marin strandavsetning, sammenhengende dekke & Marine strandvaskede sedimenter med mektighet større enn 0,5 m, dannet av bølge- og strømaktivitet i strandsonen, stedvis som strandvoller. Materialet er ofte rundet og godt sortert. Kornstørrelsen varierer fra sand til blokk, men sand og grus er vanligst. Strandavsetninger ligger som et forholdsvis tynt dekke over berggrunn eller andre sedimenter. & 42 \\ \hline
    Hav- og fjordavsetning  og strandavsetning, usammenhengende eller tynt dekke over berggrunnen & Grunnlendte områder/hyppige fjellblotninger. Tykkelsen på avsetningene er normalt mindre enn 0,5 m, men den kan helt lokalt være noe større. Det er ikke skilt mellom hav-, fjord- og strandavsetning. Kornstørrelser angis normalt ikke, men kan være alt fra leir til blokk. & 43 \\ \hline
    Skjellsand & Avsetning som i stor grad består av knuste skall av kalkutskillende organismer. Er en type av bioklastisk materiale. Kornstørrelse varierer fra nesten hele skall til sand. Det kan være ansamlet store mengder av skjellsand i umiddelbar nærhet av gode skjellvekstområder. & 44 \\ \hline
    Marin gytje & Avsetning som består av finkornig materiale, silt og leir med høyt organisk innhold. Det organiske materialet er primærprodusert i vannmassene. Marin gytje finnes i områder hvor det er liten materialtransport fra land. Brukes sjelden. & 45 \\ \hline
    Elve- og bekkeavsetning (Fluvial avsetning) & Materiale som er transportert og avsatt av elver og bekker. De mest typiske formene er elvesletter, terrasser og vifter. Sand og grus dominerer, og materialet er sortert og rundet. Mektigheten varierer fra 0,5 til mer enn 10 m. & 50 \\ \hline
    Elveavsetning, sammenhengende dekke & Materiale som er transportert og avsatt av elver og bekker. De mest typiske formene er elvesletter, terrasser og vifter. Sand og grus dominerer, og materialet er sortert og rundet. Brukes kun i spesielle tilfeller. & 51 \\ \hline
    Elveavsetning, usammenhengende/tynt dekke & Grunnlendte områder med elveavsetninger. Brukes kun i spesielle tilfeller. & 52 \\ \hline
    Flomavsetning (uspesifisert) & Brukes for spesielle sedimenter avsatt ved plutselig uttapning av bresjøer. & 53 \\ \hline
    Flomavsetning, sammenhengende & Brukes for spesielle sedimenter avsatt ved plutselig uttapning av bresjøer. & 54 \\ \hline
    Flomavsetning, usammenhengende/tynt & Brukes for spesielle sedimenter avsatt ved plutselig uttapning av bresjøer. Tykkelse mindre enn 0,5 m. & 55 \\ \hline
    Vindavsetning (Eolisk avsetning) & Flygesand med tykkelse mer enn 0,5 m. & 60 \\ \hline
    Forvitringsmateriale, ikke inndelt etter mektighet & Løsmasser dannet på stedet ved fysisk eller kjemisk nedbryting av berggrunnen. Gradvis overgang til underliggende fast fjell. Brukes når en ikke skiller mellom sammenhengende og usammenhengende dekke av denne avsetningstypen. & 70 \\ \hline
    Forvitringsmateriale, sammenhengende dekke & Løsmasser dannet på stedet ved fysisk eller kjemisk nedbryting av berggrunnen. Tykkelsen er mer enn 0,5 m. & 71 \\ \hline
    Forvitringsmateriale, usammenhengende eller tynt dekke over berggrunnen & Løsmasser dannet på stedet ved fysisk eller kjemisk nedbryting av berggrunnen. Grunnlendt område med tallrike fjellblotninger. & 72 \\ \hline
    Forvitringsmateriale, stein- og blokkrikt, dannet ved frostsprengning & Blokkhav, oftest i fjellområder. & 73 \\ \hline
    Skredmateriale, ikke inndelt etter mektighet & Avsetninger dannet ved steinsprang, fjellskred, snø- eller løsmasseskred fra bratte dalsider. Uspesifisert tykkelse. & 80 \\ \hline
    Skredmateriale, sammenhengende dekke, stedvis med stor mektighet & Avsetninger dannet ved steinsprang, fjellskred, snøskred eller løsmasseskred fra bratte dalsider. Symbol viser dominerende skredtype. Tykkelsen er mer enn 0,5 m og det er få fjellblotninger i området. & 81 \\ \hline
    Skredmateriale, usammenhengende eller tynt dekke over berggrunnen & Grunnlendte områder med avsetninger fra steinsprang, fjellskred, snø- og løsmasseskred fra bratte dalsider. Symbol viser dominerende skredtype. Tykkelse mindre enn 0,5 m. & 82 \\ \hline
    Steinbreavsetning & Steinur som inneholder/har inneholdt is og derfor er i bevegelse/har vært i bevegelse som en vanlig bre. Avsetningstypen dannes under permafrostforhold. & 88 \\ \hline
    Torv og myr (Organisk materiale) & Organisk jord dannet av døde planterester, med mektigheter større enn 0,5 m. Det skilles ikke mellom ulike torvtyper. & 90 \\ \hline
    Humusdekke/tynt torvdekke over berggrunn & Områder hvor humusdekket ligger rett på berggrunnen. Mektigheten av humusdekket er vanligvis 0,2 - 0,5 m, men kan lokalt være tykkere. Fjellblotninger opptrer hyppig innen slike områder. Fjellblotninger opptrer hyppig innen slike områder. & 100 \\ \hline
    Usammenhengende eller tynt løsmassedekke over berggrunnen, flere løsmassetyper, uspesifisert & Forskjellige sedimenter som danner et tynt eller usammenhengende dekke over berggrunnen. Denne betegnelsen brukes bare i spesielle tilfeller når en ikke velger å skille mellom ulike typer av løsmasser. & 101 \\ \hline
    Sammenhengende løsmassedekke av flere jordarter & Vanligvis skredmateriale med morenemateriale, forvitringsmateriale, torv og humus sterkt blanda ved skråningsprosesser. Brukes kun i spesielle tilfeller der det er meget vanskelig å skille mellom opprinnelige løsmassetyper. & 102 \\ \hline
    Bart fjell/fjell med tynt torvdekke, uspesifisert & Brukes når en ikke velger å skille mellom bart fjell og humusdekke eller tynt torvdekke over berggrunnen. & 110 \\ \hline
    Fyllmasse (antropogent materiale) & Løsmasser tilført eller sterkt påvirket av menneskers aktivitet, vesentlig i urbane områder. & 120 \\ \hline
    Steintipp & Tilførte steinmasser. & 121 \\ \hline
    Menneskepåvirket materiale, ikke nærmere spesifisert & Dominerende stedegne masser, omarbeidet i overflaten. & 122 \\ \hline
    Bart fjell & Brukes om områder som stort sett mangler løsmasser, mer enn 50 \% av arealet er fjell i dagen. & 130 \\ \hline
    Bart fjell/fjell med usammenhengende eller  tynt løsmassedekke & Brukes på oversiktskart der bart fjell slås sammen med alle typer tynt eller usammenhengende løsmassedekke. & 140 \\ \hline
    Marin suspensjonsavsetning & Finkornige (leire, silt) sedimenter transportert og avsatt fra suspensjon. Draperer vanligvis underliggende sedimenter eller fjell og er oftest lagdelt. & 200 \\ \hline
    Marin bunnstrømavsetning & Sedimenter som består av sand og grus transportert og avsatt fra bunnstrømmer. Dekker bunnen av undersjøiske kanaler laget av bunnstrømmer. Har ofte kryss-sjiktet og lentikulær- sjiktet indre struktur. & 201 \\ \hline
    Glasimarin avsetning & Hovedsakelig finkornige suspensjonsavsetninger (silt, leire) avsatt i nærhet av is/isbreer. Kan være påvirket av bunnstrømmer og utjevner topografien mer enn draperer. Forekommer i mektige lag i områder på kontinentalhyllen langs kysten og i fjorder & 202 \\ \hline
    Iskontaktavsetning & Sedimenter avsatt i kontakt med is. Kan være morene, glasifluvialt materiale, eller en blanding av glasialt avsatte sedimenter. Kornstørrelsen veksler mellom leire og grus alt etter hvilke prosesser som virket. & 203 \\ \hline
    Utvaskingslag & Sedimenter bestående av sand, grus og bergartsfragmenter etter at finstoffet er vasket vekk av bølger og strøm. Danner et dekkende lag over morene eller andre jordarter med stor variasjon i kornstørrelser. & 204 \\ \hline
    Glasifluvial deltaavsetning (marin) & Sedimenter transportert av breelver og avsatt i hav, bresjø eller innsjø. & 205 \\ \hline
    Fluvial deltaavsetning & Sedimenter avsatt ved utløpet av en elv i en fjord, innsjø eller i havet. Kornstørrelsen er ofte i sandfraksjonen nær elveutløpet og mer finkornig på dypere vann. Har typisk skrålaging med helling i strømretningen. & 206 \\ \hline
    Tidevannsavsetning & Avsetning dannet i kystnære områder ved tidevannstransport. Sedimentene er sandige til leirholdige med typiske strukturer som sanddyner, rifler, kryss-sjikting, mikro-kryss-sjikting, flasersjikting og lentikulær sjikting. & 207 \\ \hline
    Estuarin avsetning & Et sediment avsatt i brakkvann i et estuarie. Sedimentet er karakterisert av finkornig materiale (silt, leire) av marin og fluvial opprinnelse blandet med en høy andel rester av terrestrisk organisk materiale. & 208 \\ \hline
    Levé avsetning (marin) & Avsetning dannet som en forhøyning av sedimenter langs en eller begge sidene  av en undersjøisk kanal (kløft, viftedal eller dyphavskanal). Avsetningen kan ha varierende kornstørrelse, fra finkornig (leir) til nokså grovt materiale (sand). & 209 \\ \hline
    Grunnmarin avsetning & Sedimenter avsatt i turbulent grunt marint miljø der det fineste materialet er vasket ut og transportert til dypere vann av strømmer og bølger. Består av sand, grus og stein. I områder med mye sand kan sandbølger bygges med en karakteristisk kryss-sjikting og skrålaging. & 210 \\ \hline
    Konturittavsetning & Klastiske sedimenter transportert og avsatt av kontur-strømmer langs egga kanten. Består av fint, velsortert materiale (silt og leir). Avsetningene har vanligvis horisontal- eller kryss-sjiktning og normal- eller omvendt gradering. & 211 \\ \hline
    Turbitittavsetning & Avsetninger dannet ved sedimenttransport og utfelling fra en turbidittstrøm.  Består av materiale i kornstørrelse fra leire til sand og er ofte karakterisert ved normalgradert lagning og moderat til dårlig sortering. Finnes oftest ved foten av skråninger med stor mektighet av løse sedimenter (for eksempel langs kontinentalskråningen). & 212 \\ \hline
    Debrisstrømavsetning & Avsetning fra en flytende masse av stein, jord og slam. Den består av usortert materiale der mer enn halvparten av partiklene er større enn sandstørrelse. & 213 \\ \hline
    Undersjøisk vifteavsetning & En konisk eller vifteformet avsetning beliggende ved munningen av en undersjøisk kløft. Består for det meste av fine sedimenter (leire, silt). Viften har en finlaget indre struktur med en svak helling av lagene mot dyphavet. & 214 \\ \hline
    Kanalsavsetning & Sedimenter avsatt i en kanal. Avsetningene vil vanligvis bestå av relativt grove sedimenter (sand, grus) & 215 \\ \hline
    Dypmarin avsetning & Samlebetegnelse på dyphavssedimenter. Kan være både konturittisk, hemipelagisk, eupelagisk osv. Dette er fine sedimenter bunnfelt utenfor kontinentalmarginen. Består i stor grad av leire og rester av pelagiske organismer. & 216 \\ \hline
    Bioklastisk avsetning & Sediment som for en stor del består av små partikler av biologisk opprinnelse (skjell, korall). Kornstørrelsen kan variere fra sand til hele skjell eller korallkolonier. Forekommer i begrensete områder der vekstforholdene har vært optimale over lengre tid og mengden av annet klastisk materiale liten. & 217 \\ \hline
    Vulkanosedimentær avsetning & Avsetning som består av materiale av vulkansk opprinnelse. Alt etter kornstørrelse kan sedimentene deles inn i vulkansk aske, lapilli (2-64 mm) og breksje (>64mm). & 218 \\ \hline
    Lagdelte sedimenter (>1 m) over debrisstrøm & Lagdelte sedimenter (>1m) over debrisstrømavsetning. & 219 \\ \hline
    Skredmateriale, dekket av yngre sedimenter & Skredmateriale, dekket av yngre sedimenter & 240 \\ \hline
    Skredmateriale, delvis dekket av yngre sedimenter & Skredmateriale, delvis dekket av yngre sedimenter & 241 \\ \hline
    Skredmateriale og hemipelagiske avsetninger & Veksling mellom skredavsetninger og hemipelagiske avsetninger. Hemipelagiske avsetninger består stort sett av finkornet materiale, delvis produsert i vannmassene lokalt, og delvis tilført utenifra. & 242 \\ \hline
    Uspesifisert marin avsetning & Marin avsetning med ukjent opprinnelse. & 250 \\ \hline
    Jordskredavsetning, sammenhengende dekke, stedvis med stor mektighet & Avsetning som dannes når løsmasser i bratt terreng løsner og raser nedover. Danner ofte karakteristiske vifte- eller tungelignende former. & 301 \\ \hline
    Jordskredavsetning, usammenhengende eller tynt dekke & Grunnlendt avsetning som dannes når løsmasser i bratt terreng løsner og raser nedover. & 302 \\ \hline
    Leirskredavsetning, sammenhengende dekke, stedvis med stor mektighet & Avsetning som dannes når leirholdige sedimenter løsner og glir ut. & 303 \\ \hline
    Leirskredavsetning, usammenhengende eller tynt dekke over berggrunnen & Avsetning som dannes når leirholdige sedimenter løsner og glir ut. & 304 \\ \hline
    Fjellskredavsetning, sammenhengende dekke, stedvis med stor mektighet & Dannes når store fjellparti løsner og med kolossal kraft går ned i daler og fjorder. Består mest av kantete blokker. & 305 \\ \hline
    Fjellskredavsetning, usammenhengende eller tynt dekke & Grunnlendte områder med fjellskredmateriale. & 306 \\ \hline
    Steinsprangavsetning, sammenhengende dekke, stedvis med stor mektighet & Materiale som har løsnet fra fast fjell og over tid akkumulert som bratte urer ved foten av skråninger. Materialet varierer fra sand til blokk, med økende kornstørrelse nedover skråningen. & 307 \\ \hline
    Steinsprangavsetning, usammenhengende eller tynt dekke & Grunnlendte områder med steinsprangmateriale. & 308 \\ \hline
    Snøskredavsetning, sammenhengende dekke, stedvis med stor mektighet & Dannes i områder med gjentatte snøskred og har ofte vifteform. & 309 \\ \hline
    Snøskredavsetning, usammenhengende eller tynt dekke & Grunnlendte områder med snøskredmateriale. & 310 \\ \hline
    Fjellskred-/steinsprangavsetning, sammenhengende dekke, stedvis med stor mektighet & Materiale bestående av steinblokker fra større fjellparti som har løsnet og rast ned. Består hovedsakelig av usortert grovt materiale (stein og blokk) og finnes oftest ved foten av skrenter/fjellsider. & 311 \\ \hline
    Fjellskred-/steinsprangavsetning, usammenhengende eller tynt dekke & Grunnlendte områder med fjellskred-/steinsprangmateriale. & 312 \\ \hline
    Snø- og jordskredavsetning, sammenhengende dekke & Dannes i områder med vekslende snø- og jordskred. & 313 \\ \hline
    Snø- og jordskredavsetning, usammenhengende eller tynt dekke & Grunnlendte områder med snøskredmateriale og jordskredmateriale. & 314 \\ \hline
    Jordskred- og steinsprangavsetning, sammenhengende dekke & Dannes i bratt terreng der både jordskred og steinsprang forekommer. & 315 \\ \hline
    Jordskred- og steinsprangavsetning, usammenhengende eller tynt dekke & Grunnlendte områder med jordskred- og steinsprangmateriale. & 316 \\ \hline
    Finkornig organiskholdig sigejord & Sterkt frostpåvirket blandingsmateriale som beveger seg sakte nedover slake skråninger, dannet fra en eller flere opprinnelig finstoffholdige løsmassetyper. & 320 \\ \hline
    Steinrikt sigende skråningsmateriale & Grovkornig frostpåvirket blandingsmateriale som beveger seg sakte nedover skråninger, dannet fra forvitret fjell, skråningsmateriale eller morenemateriale. & 321 \\ \hline
\end{longtable}