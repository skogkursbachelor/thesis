\chapter{Conclusion}\label{chap:conclusion}

\begin{comment}
\textcolor{orange}{FRA PRESENTASJON OM KONKLUSJON:
    1. Gå tilbake til temaet
    2. Gjenoppgi hovedpåstanden eller gjenta problemstillingen
    3. Oppsummering av ideene som er diskutert
    4. Oppsummerende sluttpoeng:
        - Forsterke hovedbudskapet
        - Gi en tankevekker eller refleksjon
        - Peke på videre implikasjoner}    
\end{comment}

The aim of this project was to develop and test a \textbf{prototype system for fully digital modeling of forestry road load-bearing capacity under varying conditions throughout the year.} Skogkurs wanted to investigate the prototype as a proof of concept and evaluate its feasibility and effectiveness as a foundation for a future operational system in the forestry industry. The system consists of a interactive map-based website visualizing the trafficability of Norwegian forestry roads in a traffic-light system based on soil saturation, frost depth, and surrounding superficial deposits. Our proof of concept is the first to solve this problem in a way that presents trafficability data in this format, positioning it as an innovative contribution to the field. We believe this prototype has proven ... 

\textcolor{orange}{her har vi noen alternativer:}

that the concept can increase the effectiveness of route planning for transport managers.

that a digital model of load-bearing capacity can increase the effectiveness of route planning for transport managers.

to increase the effectiveness of route planning for transport managers.

\textcolor{orange}{så avsluttende:}

Given the unique solution/approach and positive feedback of the prototype, we believe that this concept should be further investigated and developed.


\textcolor{orange}{\texttt{------------------------ ELLER DELE OPP I TO AVSNITT (VI LIKER DEN UNDER MEST) ---------------------------}} 

The aim of this project was to develop and test a \textbf{prototype system for fully digital modeling of forestry road load-bearing capacity under varying conditions throughout the year.} Skogkurs wanted to investigate the prototype as a proof of concept and evaluate its feasibility and effectiveness as a foundation for a future operational system in the forestry industry. 

The system consists of an interactive, map-based website that visualizes the trafficability of Norwegian forestry roads using a traffic-light system. This classification is based on key environmental factors such as soil saturation, frost depth, and surrounding superficial deposits. To our knowledge, this is the first solution that presents trafficability in this format, making it an innovative contribution to the field. We believe the prototype has demonstrated that the concept can improve route planning efficiency for transport managers, and should therefore be further investigated and developed.


%The unique solution and positive feedback of the prototype has proved that this concept should be further investigated and developed.

%The system consists of an interactive, map-based website that visualizes the trafficability of Norwegian forestry roads using a traffic-light system. This classification is based on key environmental factors such as soil saturation, frost depth, and surrounding superficial deposits. To our knowledge, this is the first solution that presents trafficability data in this visual and accessible format, making it an innovative contribution to the field. We believe the prototype has demonstrated that the concept can improve route planning efficiency for transport managers.

%The uniqueness of the solution, along with the positive feedback it received, indicates that the concept has significant potential and should be further explored and developed.