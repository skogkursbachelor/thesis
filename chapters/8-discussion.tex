\chapter{Discussion}
\section{Process}
\subsection{Project Plan}
\subsection{Technologies}
\subsection{SDLC Model / SCRUM}
\subsection{Communication \& Group Work}
\subsection{Improvements}
\section{Product}
\subsection{Limitations}
% IKKE PROGNOSE PÅ ENKELTE KARTLAG ?! (IKKE EN UKE FREM SOM SAGT I REQUIREMENTS)
% Feilkilder (kartdata) e.g. https://www.senorge.no/WaterMap
% Vanskelig å få tilgang til satellittdata fra SMAP & SENTINENTAL-1. For å få prognose må man ha en modell for å regne ut.
\subsection{Comparison with Existing Products?} 
\subsection{Unexpected Findings} % Kanskje fjern?
\subsection{Sustainability} % KANSKJE DELER OPP I FORSKJELLIGE ASPEKTER AV BÆREKRAFT? SOFTWARE/PRODUCT, FORESTRY, etc.
\subsubsection*{Environmental}
\begin{comment}
% Står litt om skogkurs og bærekraft: https://s46339.pcdn.co/wp-content/uploads/Sluttrapport.pdf
% OG HER: https://skogkurs.no/fagartikler/baerekraftige-metoder-og-kompetanse-i-skogsmaskinbransjen-kurs-kommer/ ->
- mer lukkede hogstformer og mindre bruk av flatehogst 
- **lavere drivstofforbruk og smartere kjøring med skogsmaskin** 
    - **lavere førerbelastning**
    - **mindre terrengslitasje**
    - **høyere produktivitet**
- produsere tømmer som er mest mulig tilpasset industriens behov 
- **videreutvikle teknologi for driftsoppfølging og førerstøtte for maskinførerne** 

###########################
FNs Bærekraftsmål som virker relevante (KANSKJE SAMMENLIGN MED HVA NORGE GJØR I DAG?):
- 9
- 12 (?) VIRKER SOM DEN FOKUSERER MEST PÅ UTVIKLINGSLAND?
- 15 (?)
\end{comment}
\subsubsection*{Economic}
\subsubsection*{Social}
\subsection{Future Work}