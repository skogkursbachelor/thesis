\chapter{Deployment}\label{chap:deployment}

\textcolor{orange}{From Template: This chapter should describe how your solution can be deployed on the employer's system. It should include technical details on how to set it up, as well as discussions on choices made concerning scalability, maintenance, etc.}

\section{Docker \& Docker Compose}

\textcolor{orange}{Avsnitt hvordan vi har deploya (docker compose på vm)}

\textcolor{orange}{kanskje ha glossary på: made using the CI/CD tool GitHub Actions}.
The Docker images are made using GitHub Actions\footnote{\url{https://github.com/features/actions}}. Any commit pushed to the server or website repositories will trigger the action, which then builds the image and pushes it to GitHub Container Registry\footnote{\url{https://docs.github.com/en/packages/working-with-a-github-packages-registry/working-with-the-container-registry}}. All Docker images are built using a text document that contains all the commands a user could call on the command line to assemble an image, called a Dockerfile \cite{dockerfiledocs}. The Dockerfiles along with the GitHub Action for the website and server can be found in their respective repositories (see \autoref{tab:githubrepositories}).

\textcolor{orange}{Avsnitt om hvorfor vi valgte dette? Kanskje discussion??}

\textcolor{orange}{
Avsnitt om Hva må SKogkurs gjøre for å deploye selv på egne servere? \\
Bygge imagene selv, eller endre container reg \\
Laste ned løsmasse og fjorddata \\
Ha en maskin med docker som er åpent for nett
}


\section{OpenStack}\label{sec:deployment:openstack}

\textcolor{orange}{
Hva openstack er \\ \\
Hva SkyHiGh er
}

\subsection{Website}\label{subsec:deployment:docker:website}

\textcolor{orange}{Skrive vite?}

\subsection{Backend Server}\label{subsec:deployment:docker:backendserver}

\textcolor{orange}{Hvilke data man trenger, kanskje scrappe subsec back og website, ta med i docker og compose}
