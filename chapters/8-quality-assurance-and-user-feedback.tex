\chapter{Quality Assurance and User Feedback}\label{chap:testinganduserfeedback}

\textcolor{orange}{NOE TEKST}

\section{Quality Assurance}

\subsection{Documentation}

Proper documentation is crucial, as there is a possibility that others may use the code in the
future. To address this, both the code and the development process documented
throughout the project. 

The website is documented using JSDoc \textcolor{orange}{skriv mer om dette etter at det er gjort}.  


All tasks will be tracked and recorded as GitHub issues, ensuring a clear
history of the project’s progress. 

\subsection{Testing} %BEDRE NAVN?

\textcolor{orange}{Dette står i planen: Plan for Inspections & Testing
• Implement CI/CD pipelines on GitHub to automate processes as much as possible, starting
early in development.
• The pipelines will include:
– Static Code Analysis and Linting
– Unit and Integration Testing
– Deployment
• Conduct user testing with the Product Owner and potential end-users once the MVP is
complete. Interviews and follow-ups will be conducted to measure satisfaction and gather
feedback for the MMP.
• For unit and integration testing, we aim for 80\% or higher coverage, with a primary focus on
critical components and an emphasis on test quality over quantity.
• To monitor errors, identify potential availability and integrity issues, and facilitate debugging,
a logging system will be implemented.}

\section{User Testing}

Due to time constraints and the challenge of recruiting enough relevant participants, we were not able to conduct formal user testing during the project period. However, we recognize that user testing would be highly valuable in the future, both to improve the \acrfull{ux} and to identify additional features that transport managers would find useful in their day-to-day work.

Instead, we relied on iterative feedback from the product owner, which proved essential in guiding the development. This feedback primarily focused on functionality, particularly the need for greater flexibility for transport managers. For example, it was suggested that users should be able to adjust threshold values used in classifying trafficability, allowing them to tailor the system to their operational requirements by setting these thresholds based on their local knowledge and past experience. Feedback also covered the relevance and reliability of data sources used to assess forestry road conditions.

Based on this input, we implemented several concrete changes. Most notably, as mentioned earlier, we added the feature that allows users to customize threshold values for map layers. Additionally, we integrated a soil moisture map layer, which provides transport managers with deeper insight into road conditions and supports more informed decisions.

Toward the end of the project, we presented the prototype to two senior leaders in the forest owners’ association. Their feedback was very positive, and they expressed that the solution could be a highly useful tool for the forestry industry. During the presentation, we emphasized the system’s flexible architecture, which makes it relatively straightforward to incorporate additional map layers or data sources in the future to enhance both accuracy and functionality.

\subsection{Result}

\textcolor{orange}{NOE TEKST}

\subsection{Product Iteration and Polish}

\textcolor{orange}{NOE TEKST}
