
% From https://www.overleaf.com/learn/latex/Glossaries

\makeglossaries % Prepare for adding glossary entries


\newglossaryentry{latex}
{
        name=latex,
        description={Is a mark up language specially suited for
scientific documents}
}

\newglossaryentry{productowner}
{
        name=product owner,
        description={Single person that is accountable for maximizing the value of the product resulting from the work of the Scrum Team \cite{scrum_guide}}
}

\newglossaryentry{bibliography}
{
        name=bibliography,
        plural=bibliographies,
        description={A list of the books referred to in a scholarly work, typically printed as an appendix}
}

\newglossaryentry{frost}
{
    name=frost,
    description={Frost is frozen soil that occurs when the water in the soil (soil moisture) freezes into ice. The depth of the frost depends on the temperature at the ground surface and how thick the snow cover is \cite{senorge_terminology}}
}

\newglossaryentry{glaciofluvial deposit}
{
    name=glaciofluvial deposit,
    description={Consists mostly of sorted layers of different grain sizes, ranging from fine sand to stone and boulders, with a relatively high degree of rounding. They have high porosity and permeability \cite{ngu_deposits}}
}

\newglossaryentry{superficial deposit}
{
    name=superficial deposit,
    description={Superficial deposits are soil layers covering the solid bedrock, including stones, gravel, sand, clay, peat, and moraine material \cite{snl_losmasser}}
}

\newglossaryentry{moraine}
{
    name=moraine,
    description={A mixture of clay, silt, sand, gravel, and boulders with low or high degrees of roundness. The composition, i.e. the sorting, porosity, and permeability varies from area to area. Cracks in the moraine cover with high permeability may also occur \cite{ngu_deposits}}
}

\newglossaryentry{trafficability}
{
    name=trafficability,
    description={Reflects the minimum bearing capacity required to avoid road deformation \cite{fjeld2023trafficability}}
}

\newglossaryentry{permeability}
{
    name=permeability,
    description={The capacity of a porous material for transmitting a fluid \cite{britannica_permeability}}
}

\newglossaryentry{smap}
{
    name=SMAP,
    description={The Soil Moisture Active Passive mission is an orbiting observatory that measures the amount of water in the surface soil everywhere on Earth \cite{nasaSMAP}}
}

\newglossaryentry{sentinel-1}
{
    name=sentinel-1,
    description={Sentinel-1 is an ESA radar satellite mission providing all-weather, day-and-night Earth observation for monitoring land, soil moisture, floods, and forestry \cite{esa_sentinel-1}}
}

\newglossaryentry{geojson}
{
    name=GeoJSON,
    description={GeoJSON is a widely used format for encoding geographic data structures in JSON. It supports points, lines, polygons, and multi-geometries, making it ideal for web-based mapping applications \cite{geojson}. GeoJSON is often used for vector data in GIS applications}
}

\newglossaryentry{wms}
{
    name=WMS,
    description={Web Map Service is a standard protocol developed by the OGC for serving georeferenced map images over the internet. WMS provides dynamic map rendering based on geographic data from a server, supporting different layers and styles, often used for visualizing raster-based geospatial data \cite{ogc2006wms}}
}

\newglossaryentry{wfs}
{
    name=WFS,
    description={Web Feature Service is an OGC standard for serving vector geospatial data over the web. Unlike WMS, which delivers pre-rendered images, WFS allows clients to retrieve raw geographic features in formats like GeoJSON or GML, enabling more flexible data analysis and editing \cite{ogc2005wfs}}
}

\newglossaryentry{openstack}
{
    name=OpenStack,
    description={OpenStack is an open-source cloud computing platform for managing virtualized resources like computing, storage, and networking. It is used to build scalable private and public cloud infrastructures \cite{openstack}}
}

\newglossaryentry{openstreetmap}
{
    name=OpenStreetMap,
    description={OpenStreetMap (OSM) is a freely accessible and openly editable map database that is continuously updated and maintained by a global community of volunteers through collaborative efforts \cite{openstreetmap}}
}


% --------------------
% ----- Acronyms -----
% --------------------

\newacronym{phd}{PhD}{philosophiae doctor}
%\newacronym{CoPCSE}{CoPCSE@NTNU}{Community of Practice in Computer ScienceEducation at NTNU}
%\newacronym{gcd}{GCD}{Greatest Common Divisor}
\newacronym{nasa}{NASA}{National Aeronautics Space Administration}
\newacronym{esa}{ESA}{European Space Agency}
\newacronym{api}{API}{Application Programming Interface}
\newacronym{html}{HTML}{HyperText Markup Language}
\newacronym{sdlc}{SDLC}{Software Development Life Cycle}
\newacronym{ngu}{NGU}{Norges Geologiske Undersøkelse (Geological Survey of Norway)}
\newacronym{nve}{NVE}{Norges vassdrags- og energidirektorat (The Norwegian Water Resources and Energy Directorate)}
\newacronym{nibio}{NIBIO}{Norsk institutt for bioøkonomi (Norwegian Institute of Bioeconomy Research)}
\newacronym{swi}{SWI}{Soil Water Index}
\newacronym{gui}{GUI}{Graphical User Interface}
\newacronym{ui}{UI}{User Interface}
\newacronym{mvp}{MVP}{Minimal Viable Product}
\newacronym{mmp}{MMP}{Minimal Marketable Product}
\newacronym{ux}{UX}{User Experience}
\newacronym{susaf}{SusAF}{Sustainability Awareness Framework}
\newacronym{sdg}{SDG}{Sustainable Development Goal}